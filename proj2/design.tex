\documentclass{article}
\usepackage[utf8]{inputenc}
\usepackage{amsmath}
\usepackage{graphicx}
\usepackage[inline]{enumitem}
\graphicspath{{img/}}
\usepackage{listings}

\title{Project 2 Design Doc - CS161}
\author{yiwen song (cs161-jp), Huadian Henry Zhang (cs161-sy)}
\date{}

\begin{document}

\maketitle

\section{Design Summary}
The basic structure of our secure distributed storage system is as follows:
\begin{enumerate}
\item Userdata block to generate user-specific keys
\item File Metadata block to generate file-specific keys
\item Sharing block to generate keys for files shared with the user
\item Data block to actually store the files
\end{enumerate}
At the beginning, we have the RSA keys for the user $(K_U, K_U^{-1})$.
All non-MAC and non-Signature entries will be put into a JSON string.
Any MACs and RSA signatures used will be inputted with this JSON string.
Then the original JSON string and the MAC/Signature are put into a new JSON string and uploaded.
\\ \\
This design uses a PRNG.
This PRNG is made with AES-CTR.
Setting the IV and counter to be 0, we then input the seed as the key of this encryption algorithm.
The resulting stream of output is used as keys for other things such as encryption or MAC.

\subsection{Userdata}
\subsubsection{Naming}
The Userdata section is named \textit{username}\texttt{.userdata}.

\subsubsection{Contents}
The Userdata block contains the following:
\begin{enumerate}
\item Username
\item PRNG seed $S_U$
\item RSA Signature
\end{enumerate}

\subsubsection{Actions}
\label{sssec:userdata_actions}
The Userdata block is encrypted with asymmetric RSA using the user's public key.
The signature is signed using the user's private key.
After obtaining the Userdata block, the user will generate the following keys using the PRNG with seed $S_U$:
\begin{enumerate}
\item Encryption key for Metadata block $K_E$
\item MAC key for Metadata block $K_A$
\item Name-hash key for Metadata block $K_N$
\item Encryption key for Sharing block $K_F$
\item MAC key for Sharing block $K_B$
\end{enumerate}
We should also check that the username matches and use verify on the signature to make sure that the data is
not tampered with.

\subsection{Metadata}
\subsubsection{Naming}
Let $s = username\texttt{.}filename\texttt{.meta}$.
Using $K_N$ found in \ref{sssec:userdata_actions}, we find $m = \text{SHA256-HMAC}_{K_N}(s)$.
This file will be stored at $m$.

\subsubsection{Contents}
The Metadata block contains the following:
\begin{enumerate}
\item Filename
\item PRNG seed $S_M$
\item Number of Blocks
\item Sharing list
\item MAC
\end{enumerate}

\subsubsection{Actions}
\label{sssec:meta_actions}
The Metadata block is encrypted with AES-CBC using $K_E$.
The MAC of the Metadata block is calculated using $K_A$.
(Keys are from \ref{sssec:userdata_actions})
The number of blocks is calculated by taking the ceiling of the filesize divided by the block size,
which is set to be 32 KB.
From $S_M$, we should use the PRNG to generate the following keys:
\begin{enumerate}
\item Encryption key for Data block $K_G$
\item MAC key for Data block $K_C$
\item Name-hash key for Data block $K_M$
\end{enumerate}
When the user gets the block from the server, the two checks that occur are that the MAC is valid and that
the filename matches the filename that the user is attempting to access.
\\ \\
Whenever we add a user as a collaborator on a file, we generate a random seed and copy information from
this Metadata block (more about this in \ref{ssec:sharing}) to the file corresponding to that seed.
Then, we add the tuple $(username,randomname)$ to the sharing list.
\\ \\
To remove collaborators, we have to generate an entirely new Metadata block and copy all contents of a file
to a new location.
Then we remove the tuple from the Sharing list.

\subsection{Sharing}
\label{ssec:sharing}
\subsubsection{Naming}
The sharing block for a user is named $username\texttt{.sharing}$.

\subsubsection{Contents}
\begin{enumerate}
\item Username
\item Shared list
\item MAC
\end{enumerate}

\subsubsection{Actions}
The Sharing block is encrypted with $K_F$
and MAC with $K_B$ (both from \ref{sssec:userdata_actions}).
The shared list contains tuples in the format $(username, filename, S_F)$.
For sharing, we can generate 3 keys from the seed:
\begin{enumerate}
\item Key for encryption $K_{SE}$
\item Key for MAC $K_{SA}$
\item Key for naming $K_{SN}$
\end{enumerate}
The Metadata block will be stored using the same scheme as the regular Metadata storage, and this block
can be accessed as though the file belongs to the user that is shared with.
When a user receives a sharing message, he should decrypt that message with his private RSA key and add
the tuple to the Shared list in this block.
Similarly, when a user shares any file, the share should send a message that is this tuple encrypted with the
receiving user's public RSA key.

\subsection{Data}
\subsubsection{Naming}
Let $s = username\texttt{.}filename\texttt{.}n\texttt{.data}$ where $n$ is the block number.
Using $K_M$ found in \ref{sssec:meta_actions}, we find $m = \text{SHA256-HMAC}_{K_M}(s)$.
This file will be stored at $m$.

\subsubsection{Contents}
File name and block numbers are stored to check for block swapping by adversaries.
32KB of data is stored.
Each block has a randomly generated salt (generated every time a block is made) to prevent adversaries from
being able to compare two blocks with the same information and knowing what they are.
Finally, there is a MAC to make sure that nothing is tampered with.

\subsubsection{Actions}
This block is encrypted with $K_G$ and the MAC uses $K_C$, both from \ref{sssec:meta_actions}.
The MAC should be checked before the data is used.
When the data is received, the file name and block number should also be checked to make sure that they
match what is expected.
When all the Data blocks are downloaded and verified for integrity, simply piece the data in block order
and return the string.

\end{document}






















